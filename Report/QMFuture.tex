\section{Future of Quantum Machine Learning}
% Quantum algorithms can either \emph{assist} the ML model or replace them entirely with quantum computing techniques.
In \ref{sec:qalse}, we saw a QC technique replacing the entire model. Though it is way more efficient and scalable than classical methods, it is \emph{not feasible} in near future where a quantum device will contain 100-1000 qubits. Even after the exponential speed-up, large-scale models will require millions of qubits and so these algorithms have no practical value (atleast in the near-term).

So for near-term quantum computers, the best way to proceed is for quantum computers to \emph{assist} classical ML models. This is called as Quantum-Assisted Machine Learning (QAML).

There are three things where QAML can focus according to \cite{Perdomo_Ortiz_2018} that has the potential for killer applications
\begin{itemize}
\item Hard and out of scope problems for ML. E.g., generative models for semi-supervised and unsupervised ML
\item Datasets with quantum-like connections, turning quantum computers absolutely necessary. E.g., cognitive sciences
\item Hybrid algorithms with a difficult execution step for ML pipeline.
\end{itemize}