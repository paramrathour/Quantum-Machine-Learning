% Copyright 2004 by Till Tantau <tantau@users.sourceforge.net>.
%
% In principle, this file can be redistributed and/or modified under
% the terms of the GNU Public License, version 2.
%
% However, this file is supposed to be a template to be modified
% for your own needs. For this reason, if you use this file as a
% template and not specifically distribute it as part of a another
% package/program, I grant the extra permission to freely copy and
% modify this file as you see fit and even to delete this copyright
% notice. 

\documentclass[aspectratio=169, handout]{beamer}

% There are many different themes available for Beamer. A comprehensive
% list with examples is given here:
% http://deic.uab.es/~iblanes/beamer_gallery/index_by_theme.html
% You can uncomment the themes below if you would like to use a different
% one:
%\usetheme{AnnArbor}
%\usetheme{Antibes}
%\usetheme{Bergen}
%\usetheme{Berkeley}
%\usetheme{Berlin}
\usetheme{Boadilla}
%\usetheme{boxes}
%\usetheme{CambridgeUS}
%\usetheme{Copenhagen}
%\usetheme{Darmstadt}
%\usetheme{default}
%\usetheme{Frankfurt}
%\usetheme{Goettingen}
%\usetheme{Hannover}
%\usetheme{Ilmenau}
%\usetheme{JuanLesPins}
%\usetheme{Luebeck}
% \usetheme{Madrid}
%\usetheme{Malmoe}
%\usetheme{Marburg}
%\usetheme{Montpellier}
%\usetheme{PaloAlto}
%\usetheme{Pittsburgh}
%\usetheme{Rochester}
%\usetheme{Singapore}
%\usetheme{Szeged}
%\usetheme{Warsaw}

\title[Quantum Machine Learning]{Quantum Machine Learning: Opportunities and Challenges}
% A subtitle is optional and this may be deleted
\subtitle{EE350 Technical Communication}

% \author{F.~Author\inst{1} \and S.~Another\inst{2}}
\author[Param]{Rathour Param Jitendrakumar\\190070049}
% - Give the names in the same order as the appear in the paper.
% - Use the \inst{?} command only if the authors have different
%   affiliation.

\institute[IIT Bombay]{Department of Electrical Engineering\\
Indian Institue of Technology Bombay} % (optional, but mostly needed)
% {
%   \inst{1}%
%   Department of Computer Science\\
%   University of Somewhere
%   \and
%   \inst{2}%
%   Department of Theoretical Philosophy\\
%   University of Elsewhere}
% - Use the \inst command only if there are several affiliations.
% - Keep it simple, no one is interested in your street address.

\date{Spring 2021-22}
% - Either use conference name or its abbreviation.
% - Not really informative to the audience, more for people (including
%   yourself) who are reading the slides online

\subject{Technical Communication}
% This is only inserted into the PDF information catalog. Can be left
% out. 

% If you have a file called "university-logo-filename.xxx", where xxx
% is a graphic format that can be processed by latex or pdflatex,
% resp., then you can add a logo as follows:

% \pgfdeclareimage[height=0.5cm]{university-logo}{university-logo-filename}
% \logo{\pgfuseimage{university-logo}}

% Delete this, if you do not want the table of contents to pop up at
% the beginning of each subsection:
% \AtBeginSubsection[]
% {
%   \begin{frame}<beamer>{Outline}
%     \tableofcontents[currentsection,currentsubsection]
%   \end{frame}
% }
\newcommand{\bF}{\mathbb{F}}
\newcommand{\bV}{\mathbb{V}}
\newcommand{\bU}{\mathbb{U}}
\newcommand{\bW}{\mathbb{W}}
\newcommand{\bR}{\mathbb{R}}
\newcommand{\floor}[1]{\lfloor #1 \rfloor}
\newcommand{\qbit}{\ensuremath|\psi\rangle}
\newcommand{\qbitv}{\ensuremath|\varphi\rangle}
\newcommand{\qz}{\ensuremath|0\rangle}
\newcommand{\qo}{\ensuremath|1\rangle}
\renewcommand{\d}{\, \mathrm{d}}
\theoremstyle{example}
\newtheorem{postulate}{Postulate}
\usepackage{braket}
\usepackage{cite}
\usepackage{amsmath,amssymb,amsfonts,mathtools,nccmath}
\usepackage{algorithmic}
\usepackage{graphicx}
\usepackage{textcomp}
\usepackage{xcolor}
\usepackage{float}
\usepackage{ragged2e}
% \usepackage{etoolbox}
% \apptocmd{\frame}{}{\justifying}{} % Allow optional arguments after frame.
\renewcommand{\raggedright}{\leftskip=0pt \rightskip=0pt plus 0cm}
\apptocmd{\frame}{}{\justifying}{}
% \addtobeamertemplate{}{}{\justifying}
\setbeamersize{text margin left=2em,text margin right=3em}
% \beamerdefaultoverlayspecification{<+->}
% \addtobeamertemplate{proof begin}{%
%     \setbeamercolor{block title}{fg=red!50!black,bg=red!25!white}%
%     \setbeamercolor{block body}{fg=black, bg=red!10!white}%
% }{}
% Let's get started
\begin{document}

\begin{frame}
  \titlepage
\end{frame}

\begin{frame}{Outline}
  \tableofcontents
  % You might wish to add the option [pausesections]
\end{frame}

% Section and subsections will appear in the presentation overview
% and table of contents.
\section{Machine Learning}

\begin{frame}{Machine Learning}
  \begin{itemize}
  \pause\item {
    Machine Learning (ML) is the practice of estimating models that make predictions on `new' data using available data. 
  }
  \pause\item {
    Helpful when the actual model is very complex and no known algorithms can discover such model or they take large amount of time or other resources.
  }
  \end{itemize}
\end{frame}
\subsection{Machine Learning Paradigms}
\begin{frame}{Machine Learning Paradigms}
  \begin{itemize}
  \pause\item {
    Supervised ML
    \begin{itemize}
    \pause\item {Classification
    \begin{itemize}
    \pause\item {Categorical (qualitative)
    }
    \pause\item {Numerical (quantitative)
    }
    \end{itemize}
    }
    \pause\item {Regression
    }
    \end{itemize}
  }
  \pause\item {Unsupervised ML
    \begin{itemize}
    \pause\item {Clustering
    }
    \pause\item {Association
    }
    \end{itemize}
  }
  \pause\item {Semi-Supervised ML
  }
  \pause\item {Reinforcement Learning
  }
  \end{itemize}
\end{frame}
\subsection{Issues with ML}
\begin{frame}{Issues with ML}
\begin{itemize}
\pause\item { Lack of good quality data sets.}
\pause\item { Possibilty of errors e.g., underfitting or overfitting possible.}
\pause\item { Result are sensitive to small pertubations.}
\pause\item { Safety and Privacy concerns exists.}
\pause\item { Lack of explainability of model.}
\pause\item { Slow process. Training large data sets eates up time.}
\end{itemize}
\end{frame}
\section{Quantum Mechanics for Quantum Computation}
\begin{frame}{Quantum Mechanics for Quantum Computation}
  \begin{itemize}
  \pause\item {
   A \emph{qubit} is the fundamental data unit of a quantum computer. 
  }
  \pause\item {
   A qubit $\qbit$ coexists in the states $\qz$ and $\qo$.
  }
  \pause\item {
   Physical Interpretation
  }
  \begin{equation}
\qbit = \alpha\qz + \beta\qo.
\end{equation}
\pause\item {
   $\qz$ and $\qo$ are unit vectors and form an orthonormal basis.
  }
\begin{equation*}
\qz = \begin{bmatrix}1\\0\end{bmatrix} \quad \text{and} \quad \qo = \begin{bmatrix}0\\1\end{bmatrix}.
\end{equation*}
 \pause\begin{definition}[Inner Product] \[\braket{\psi|\varphi} = \begin{bmatrix}\alpha^* & \beta^*\end{bmatrix}\begin{bmatrix}\gamma\\\delta\end{bmatrix}=\alpha^*\gamma+\beta^*\delta\]
\end{definition}
  \end{itemize}
\end{frame}
\subsection{Postulates of Quantum Mechanics}
\begin{frame}{Postulates of Quantum Mechanics}
\pause\begin{definition}[Hilbert Space]
A vector space with distance function and inner product among it's elements.
\end{definition}
\pause\begin{definition}[State Space]
The Hilbert space corresponding to an isolated physical system.
\end{definition}
\pause\begin{definition}[State Vector]
The state of the system.
\end{definition}
\pause\begin{postulate}[\textbf{State Space}]
The state vector can be completely represented by a unit vector in the state space.
\end{postulate}
\end{frame}
\begin{frame}{Postulates of Quantum Mechanics}
% \pause\begin{definition}[Hermitian adjoint]
% $H$ and $H^*$ are Hermitian conjugates of each other when 
% \[\langle \Phi | H | \Psi \rangle  = \langle H^* \Phi | \Psi \rangle.\]
% \end{definition}
\pause\begin{definition}[Hermitian operator]
A Hermitian operator $H$ is its own Hermitian conjugate, i.e $H = H^*$.
\end{definition}
\pause\begin{definition}[Observable]
A Hermitian operator $\mathcal{O}$ which describes the projective measurement. It acts upon the state space of the system to be measured and is represented as 
\[\mathcal{O} = \sum_{\lambda}\lambda P_\lambda\]
and $P_\lambda$ is the \emph{projector} with eigenvalue $\lambda \in$ eigenspace of $\mathcal{O}$.
\end{definition}
\end{frame}
\begin{frame}{Postulates of Quantum Mechanics}
  \pause\begin{definition}[Unitary Transformation]
A tranformation that preserves inner product. So, a unit vector remains a unit vector after the tranformation. It is denoted by $U$ and it satisfies
\[U^*U = UU^* = I.\]
\end{definition}
\pause\begin{postulate}[\textbf{Measurement}]
A measurement on $\mathcal{O}$ will result in the eigenvalues $\lambda$ with the probability of collapsing to $\lambda$ is \[p_\lambda = \|P_\lambda\bra{\psi}\|^2 = \braket{\psi|P_\lambda|\psi}.\]
resulting in the irreversibly collapse of the state of system to
\[\frac{1}{\sqrt{p_\lambda}P_\lambda\bra{\psi}}.\]
\end{postulate}
\end{frame}

\begin{frame}{Postulates of Quantum Mechanics}
  \pause\begin{postulate}[\textbf{Evolution I}]
The system evolution is governed by the Schrödinger's equation
\[H\bra{\psi(t)}=i\hbar\frac{\d}{\d t}\bra{\psi(t)}.\]
\end{postulate}
\pause\begin{postulate}[\textbf{Evolution II}]
If $\bra{\psi_0}, \bra{\psi_1}$ describes the state of system at times $t_0, t_1$ respectively then the evolution of system depends only on $t_0$ and $t_1$. It is given by
\[\bra{\psi_1} = U\bra{\psi_0}.\]
\end{postulate}
So far, we have studied one-qubit systems. For multi-qubit systems we introduce the following notation.
\end{frame}
\begin{frame}{Postulates of Quantum Mechanics}
    \pause\begin{definition}[Tensor Product (aka Kronecker Product)]
For a given $m,n$ dimensional vector spaces $V,W$. The tensor product of $V$ with $W$ (denoted by $V\otimes W$) is an $mn$ dimensional vector spaces with elements are linear combinations of tensor products $\bra{v}\otimes\bra{w}$.
\end{definition}
We can also define the tensor product of linear operators $A,B$ over vector spaces $V,W$ as
\[(A\otimes B)(\bra{v}\otimes\bra{w})= A\bra{v}\otimes B\bra{w}.\]
\begin{postulate}[\textbf{Composition}]
For $n$ isolated systems with states $\bra{\psi_0}, \bra{\psi_1}, \ldots, \bra{\psi_{n-1}}$, the state of the composite system is given by $\bra{\psi_0}\otimes\bra{\psi_1}\otimes\cdots\bra{\psi_{n-1}}$
\end{postulate}
\end{frame}
\section{Quantum Machine Learning (QML)}
\subsection{Future of QML }
\begin{frame}{Quantum Machine Learning (QML)}{Quantum-Assisted Machine Learning (QAML)}
    \begin{itemize}
        \pause\item{Quantum Models \emph{not feasible} in near future.
    }
    \pause\item{A near-term future quantum device will contain 100-1000 qubits.
    
    }
    \pause\item
    Possible applications
    \pause
    \begin{itemize}
\pause\item Hard and out of scope problems for ML. E.g., generative models for semi-supervised and unsupervised ML
\pause\item Datasets with quantum-like connections, turning quantum computers absolutely necessary. E.g., cognitive sciences
\pause\item Hybrid algorithms with a difficult execution step for ML pipeline.
\end{itemize}
    \end{itemize}
\end{frame}
\subsection{Opportunities in Quantum-Assisted Machine Learning}
\begin{frame}{Opportunities in Quantum-Assisted Machine Learning}{Unsupervised Learning - Quantum Devices for Sampling}
  \begin{itemize}
  \pause\item {
     Incredible amount of unlabelled data available.
  }
   \pause\item {
     There is a need to extract patterns within such data
  }
   \pause\item {
  Scientists do not always know what patterns look for
  }
   \pause\item {
  A need arises for a machine which is capable of extracting order from disorder.
  }
  
  
  \end{itemize}
\end{frame}
\begin{frame}{Generative Models}
  \begin{itemize}
  \pause\item {
   Generative Model can learn the joint probability among the variables. 
  }
  \pause\item {
    If this is achieved, similar data as the training set can be generated.
  }
  \pause\item {
    Positive results for high-dimensional data, with correct inference of multi-modal distributions over it.
  }
  \pause\item {
   The intractable step is computation of expectation values under a complex distribution and this step is a part of each iteration and data point for which Markov chain Monte Carlo (MCMC) techniques are being used.
  }
  \pause\item {
    Quantum Gibbs distributions are such alternative to MCMC
  }
  \end{itemize}
\end{frame}
\begin{frame}{Cognitive Sciences - Exploiting Datasets}
  \begin{itemize}
  \pause\item {
    A quantum model can significantly reduce the computational resources, e.g, memory needed to model a data set.
  }
    \pause\item {
    Real-life data sets where quantum model is simple compared with classical models must be identified. 
  }

  \end{itemize}
\end{frame}
\begin{frame}{Cognitive Sciences - Exploiting Datasets}{The Two-Stage Gambling Paradigm}
  \begin{itemize}
  \pause\item {
    In the first stage, people were required to gamble. If they win they get $x$ amount of money, if they lose they lose $y$ amount of money (both are equally likely).
  }
    \pause\item {
     Before learning about the results, participants were asked whether they will `plan' to gamble again (if they win and if they lose).
  }
  \pause\item {
    Then after learning the first stage results, final decision was made about playing the second stage was made.
  }
  \pause\begin{table}[H]
\centering
\begin{tabular}{cccccc}
\hline
\multicolumn{2}{c}{Gamble} & \multicolumn{4}{c}{Choice Proportions} \\ \hline
Win & Loss & Plan Win & Plan Loss & Final Win & Final Loss \\ \hline
0.8 & 1 & 0.25 & 0.26 & 0.2 & 0.35 \\
0.8 & 0.4 & 0.76 & 0.72 & 0.69 & 0.73 \\
\multicolumn{1}{c}{2} & \multicolumn{1}{c}{1} & \multicolumn{1}{c}{0.68} & \multicolumn{1}{c}{0.68} & \multicolumn{1}{c}{0.6} & \multicolumn{1}{c}{0.75} \\
2 & 0.4 & 0.84 & 0.86 & 0.76 & 0.89\\\hline
\end{tabular}
\caption{The Two-Stage Gambling Paradigm\footnote{\tiny J. Busemeyer, Z. Wang, and R. Shiffrin, “Bayesian model comparison favors quantum over standard
decision theory account of dynamic inconsistency,” Decision, vol. 2, pp. 1–12, 01 2015.}}
\label{tab:ttsgp}
\end{table}
  \end{itemize}
\end{frame}
\begin{frame}{The Two-Stage Gambling Paradigm}{The Quantum Model}
  \begin{itemize}
  \pause\item Using quantum theory, this game has 4 events $\{WA, WR, LA, LR\}$ where $W (L)$ stands for win (lose) in first stage and $A (R)$ stands for accept (reject) the second round. 
  \pause\item The person is actually in a superposition of these states \[\qbit = \psi_{WA}\ket{WA} + \psi_{WR}\ket{WR} + \psi_{LA}\ket{LA} + \psi_{LR}\ket{LR}\] where $|\psi_{WA}|^2$ is the probability that the person belives he won in the first stage and will accept the second stage.
  \pause\item The initial state is $\psi_0$ which has some distribution over these 4 amplitudes.
  \pause\item \emph{Uncertainty} in the first stage results is solved at the second stage after learning the result. Now that state is $\psi_1 = \psi_W = \ket{1100}$ ($\psi_L = \ket{0011}$) if win (lose). 
  \pause\item The payoffs can be achieved using a unitary matrix which rotates $\psi$ towards the gamble or away from it. The final state is \[\psi_D = U\psi\]% \quad \text{where} \quad U = u()
  \end{itemize}
\end{frame}
\subsection{Challenges in Quantum-Assisted Machine Learning}
\begin{frame}{Challenges in Quantum-Assisted Machine Learning}
  \begin{itemize}
  \pause\item {Compatibility Issues in Hybrid Tech
  }
  \pause\item {Robustness to Noise
  }
  \pause\item {The Curse of Limited Connectivity
  }
  \pause\item {Complex Dataset Representation
  }
  \end{itemize}
\end{frame}
\begin{frame}{Quantum-Assisted Helmholtz Machine (QAHM)}
  \begin{itemize}
  \pause\item {
    A hybrid quantum–classical ML which can potentilly handle real-world datasets.
  }
  \pause\item {
    Its made of a  generator network and a recognition network using the notion of stochastic hidden variables. 
  }
  \pause\item {
   Classification can also be implemented in QAHM.
  }
  \end{itemize}
\end{frame}

% All of the following is optional and typically not needed. 
\appendix
\section<presentation>*{\appendixname}
\subsection<presentation>*{For Further Reading}
% \begin{frame}[allowframebreaks]
\begin{frame}
  \frametitle<presentation>{For Further Reading}
%   \bibliographystyle{plainurl}    
\bibliographystyle{ieeetr}
  \nocite{*}
  {\bibliography{references}}
%   \begin{thebibliography}{10}
    
%   \beamertemplatebookbibitems
%   % Start with overview books.

%   \bibitem{Author1990}
%     A.~Author.
%     \newblock {\em Handbook of Everything}.
%     \newblock Some Press, 1990.
 
    
%   \beamertemplatearticlebibitems
%   % Followed by interesting articles. Keep the list short. 

%   \bibitem{Someone2000}
%     S.~Someone.
%     \newblock On this and that.
%     \newblock {\em Journal of This and That}, 2(1):50--100,
%     2000.
%   \end{thebibliography}
\end{frame}

\end{document}


